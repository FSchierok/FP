\section{Auswertung}
\label{sec:Auswertung}


\subsection{Kontrast}

Zunächst wurde der Kontrast des Interferometers bestimmt. Dazu wurden die Minima und Maxima der Intensität in Abhängigkeit vom Winkel $\theta$ vermessen. Nach Formel \eqref{eq:kontrast} werden die Kontraste bestimmt.  Um das Maximum zu ermitteln, wird eine Funktion der Form
\begin{equation}
  K(\theta)=a\sin(2(\theta+b))
\end{equation}
an die Messwerte gefittet. Dabei ergibt sich für die Fittparameter:
\begin{align}
  a&=\num{0,88} & b&=\SI{-3,29}{°}
\end{align}
Das Maximum befindet sich also bei $\theta=\SI{46.6}{°}$, für die weiteren Messungen wurde jedoch der maximale Messwert $\theta=\SI{52.5}{°}$ verwendet, da der Fit erst nach der Durchführung des Experiments erstellt wurde.
Die Messwerte sind in Tabelle \ref{tab:kontrast} zu finden und die  Kontraste sowie der Fit sind in Plot \ref{img:kontrast} zu sehen.
\begin{figure}
  \centering
  \includegraphics[width=0.8\linewidth]{img/kontrast.pdf}
  \caption{Plot der bestimmen Kontrastwerte in Abhängigkeit von $\theta$.}
  \label{img:kontrast}
\end{figure}

\input{tab/kontrast.tex}


\subsection{Glas}

Um den Brechungsindex der Glasplatten zu bestimmen, wurden fünf Messungen vorgenommen. Die Dicke der verwendeten Glasplatte ist $T=\SI{1}{\milli\meter}$ \cite{V64}.
Mit diesen gemessenen Werten für $M$ und $\Theta$ und der Formel \eqref{eq:glasfringes} wird $n_\text{Glas}$ bestimmt. Die Messwerte und die einzeln bestimmten Brechungsindizes
 $n_\text{Glas}$ sind in Tabelle \ref{tab:glas} zu finden.
Über alle Messreihen gemittelt ergibt sich $n_\text{Glas}=\num{1,47\pm 0,02}$.

\input{tab/glas.tex}


\subsection{Luft}

Zur Bestimmung des Brechungsindexes von Luft wurde eine Gaszelle bestmöglich evakuiert und mit Luft kontrolliert wieder auf Umgebungsdruck gebracht. Die Gaszelle hat dabei eine Länge von $L=\SI{100\pm 0.1}{\milli \meter}$. Es wurde ein
Vakuum von $\SI{14}{\milli \bar}$ erreicht. Die aufgenommen Messwerte und die nach Formel \eqref{eq:gasfringes} bestimmten Brechungsindizes sind in Tabelle \ref{tab:luft} zu finden.
Gemittelt über die drei Messreihen ergibt sich $n_\text{Luft}=\num{1.0002616 \pm 0.0000002}$.
\input{tab/luft.tex}
