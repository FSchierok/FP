\section{Auswertung}
\label{sec:Auswertung}
\subsection{Kontrast}
Zunächst wurde der Kontrast des Aufbaus bestimmt. Dazu wurden die Minima und Maxima der Intensität in Abhänigkeit zum Polarisationswinkel aufgenommen. Die Daten sind in Tabelle \ref{tab:kontrast} zu finden. Die nach Formel \eqref{eq:kontrast} bestimmten Kontrastwerte sind in Plot \ref{img:kontrast} zu sehen. Das  Maximum befindet sich bei $\SI{52.5}{°}$. Diese Einstellung wurde bei allen weiteren Messungen verwendet.
\begin{figure}
  \centering
  \includegraphics[width=0.8\linewidth]{img/kontrast.pdf}
  \caption{Plot der bestimmen Kontrastwerte in Abhänigkeit von $\varphi$.}
  \label{img:kontrast}
\end{figure}
\input{tab/kontrast.tex}

\subsection{Glas}
Um den Brechungsindex des Glases zu bestimmen, wurden fünf Messungen aufgenommen. Die Dicke der verwendeten Glasplatte ist $T=\SI{1}{\milli\meter}$.
Mit diesen gemessenen $M$- und $\Theta$-Werte und der Formel \eqref{eq:glasfringes} wird $n_\text{Glas}$ bestimmt. Die Messwerte und die einzeln bestimmten
 $n_\text{Glas}$ sind in Tabelle \ref{tab:glas} zu finden.
Über alle Messreihen gemittelt ergibt sich $n_\text{Glas}=\si{1.474\pm 0.021}$.
\input{tab/glas.tex}
\subsection{Luft}
Zur Bestimmung des Brechungsindexes von Luft wurde eine Gaszelle bestmöglich evakuiert und mit Luft kontrolliert wieder auf Umgebungsdruck gebracht. Die Gaszelle hat dabei eine Länge von $L=\SI{100\pm 0.1}{\milli \meter}$. Es wurde ein
Vakuum von $\SI{14}{\milli \bar}$ erreicht. Die aufgenommen Messwerte und die nach Formel \eqref{eq:gasfringes} bestimmten $n$-Werte sind in Tabelle \ref{tab:luft} zu finden.
\input{tab/luft.tex}
Gemittelt über die drei Messreihen ergibt sich $n_\text{Luft}=\si{1.00026 \pm 0.00000}$.
