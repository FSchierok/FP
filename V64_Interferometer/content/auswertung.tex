\section{Auswertung}
\label{sec:Auswertung}
\subsection{Glas}
Um den Brechungsinfex des Glases zu bstimmen, wurden fünf Messungen aufgenommen. Die Dicke der verwendeten Glasplatte ist $T=\SI{100\pm 0,1}{\milli\meter}$
Mit diesen gemessenen $M$- und $Theta$-Werte und der Formel \eqref{eq:glasfringes} wird $n_\text{Glas}$ bestimmt. Die Messwerte und die einzeln bestimmten
 $n_\text{Glas}$ sind in Tabelle \ref{tab:glas} zu finden.
Über alle Messreihen gemittelt ergibt sich $n_\text{Glas}=\si{1,47400\pm 0,02100}$.
\input{tab/glas.tex}
\subsection{Luft}
Zur Bestimmung des Brechungsindex von Luft wurden eine Gaszelle bestmöglich evakuiert und mit Luft wieder auf Umgebungsdruck aufgefüllt. Dabei wurde ein
Unterdruck von $\SI{14}{\milli \bar}$ erreicht. Die aufgenommen Messwerte und die nach Formel \eqref{eq:gasfringes} bestimmten $n$-Werte sind in Tabelle \ref{tab:luft} zu sehen.
\input{tab/luft.tex}
Gemittelt über die drei Messreihen ergibt sich $n_\text{Luft}=\si{1,00026 \pm 0,00000}$.
