\section{Auswertung}
\label{sec:Auswertung}
\subsection{Glas}
Um den Brechungsinfex des Glases zu bstimmen, wurden fünf Messungen aufgenommen. Die Dicke der verwendeten Glasplatte ist $T=\SI{100\pm 0.1}{\milli\meter}$
Mit diesen gemessenen $M$- und $Theta$-Werte und der Formel \eqref{eq:glasfringes} wird $n_\text{Glas}$ bestimmt. Die Messwerte und die einzeln bestimmten
 $n_\text{Glas}$ sind in Tabelle \ref{tab:glas} zu finden.
Über alle Messreihen gemittelt ergibt sich $n_\text{Glas}=\num{1.474\pm 0.021}$.
\input{tab/glas.tex}
\subsection{Luft}
Zur Bestimmung des Brechungsindex von Luft wurden die Anzahl der Infernezmaxima $M$ aufgenommen.
