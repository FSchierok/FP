\section{Diskussion}
Das Kontrastmaximum befand sich bei $\SI{52.5}{°}$. Das es bei $\SI{45}{°}$ liegen sollte, wurde wahrscheinlich der Laser in seiner Halterung verdreht. Das hatte aber kein Einfluss auf die weiteren Messungen. Der Kotrast lag bei $0,89$.
Der Brechungsindex von Glas wurde mit $n_\text{Glas}=\SI{1.474 \pm 0.0021}$ bestimmt. Leider ist die Glasvariante aus dem Aufbau nicht bekannt, typische
 Werte sind aber zwischen $\si{1.45}$ und $\SI{1.51}$ \cite{filmetrics}. Der gemessene Wert ist also sehr realistisch.
Der Brechungsindex von Luft wurde auf $n_\text{Luft}=\SI{1.00026 \pm 0.00000}$ bestimmt. Der Literaturwert ist $n_\text{Lit}=\SI{1.000262}$ \cite{spektrum}. Hier kann von
einem vollen Erfolg gesprochen werden. Bei beiden Werten fällt die erstaunlich hohe Genauigkeit auf, was für die verwendete Messmethode spricht. Die Genauigkeit der Messung rechtfertigt den Aufwand der Justage, zudem können sowohl gasförmige als auch feste Proben untersucht werden. Das Sagnac-Interferometer ist also eine sinnvolle Alternative des Michelson-Interferometer.
\label{sec:Diskussion}
