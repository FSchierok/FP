\section{Diskussion}
Der Brechungsindex von Glas wurde mit $n_\text{Glas}=\si{1,47400\pm0,002100}$ bestimmt. Leider ist die Glasvariante aus dem Aufbau nicht bekannt, typische
 Werte sind aber zwischen $\si{1,145}$ und $\si{1.151}$ \cite{filmetrics}. Unser Wert ist also sehr realistisch.
Der Brechungsindex von Luft wurde auf $n_\text{Luft}=\si{1,00026 \pm 0,00000}$ bestimmt. Der Literaturwert ist $n_\text{Lit}=\si{1,000262}$ \cite{spektrum}. Hier kann von
 einem vollen Erfolg gesprochen werden. Bei beiden Werten fällt die Genauigkeit auf. Diese ist erstaunlich hoch. Dies spricht für den Aufbau. Dieser ist zwar
aufwendig im Aufbau und Justage, die Genauigkeit der Messung rechtfertigt den Aufwand aber. Zudem kann sowohl gasförmige als auch feste Proben untersucht werden.
\label{sec:Diskussion}
