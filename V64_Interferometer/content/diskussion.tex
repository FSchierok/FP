\section{Diskussion}
Das Kontrastmaximum befand sich bei $\theta = \SI{52.5}{°}$. Da es bei $\theta = \SI{45}{°}$ liegen sollte, wurde wahrscheinlich der Laser in seiner Halterung verdreht. Dies hatte jedoch keinen Einfluss auf die folgenden Messungen. Der Kontrast im Maximum lag bei $K_\text{max} = 0,89$.
Der Brechungsindex von Glas wurde zu
$n_\text{Glas}=\num{1,47 \pm 0,02}$
bestimmt.
Leider ist die verwendete Glassorte nicht bekannt,
typische Werte liegen aber zwischen $\num{1,45}$ und
 $\num{1,51}$ \cite{filmetrics}. Der gemessene Wert ist also realistisch.
Der Brechungsindex von Luft wurde zu $n_\text{Luft}=\num{1,000262 \pm 0,000000}$ bei $T=\SI{24}{°}$ und $p=\SI{1014}{\milli \bar}$ bestimmt. Der Literaturwert ist $n_\text{Lit}=\num{1,000262}$ \cite{spektrum}. Hier kann von
einem vollen Erfolg gesprochen werden. Bei beiden Werten fällt die erstaunlich hohe Genauigkeit auf, was für die verwendete Messmethode spricht. Die Genauigkeit der Messung rechtfertigt den Aufwand der Justierung, zudem können sowohl gasförmige als auch feste Proben untersucht werden. Das Sagnac-Interferometer ist also für die durchgeführten Messungen eine sinnvolle Alternative zum Michelson-Interferometer.
\label{sec:Diskussion}
