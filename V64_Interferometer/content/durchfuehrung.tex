% !TEX root = /home/carlo/Dokumente/Studium/FP/V64_Interferometrie/main.tex


\section{Durchführung}


\subsection{Messung des Kontrasts}
Zur Kontrastmessung wird ein zweiter Polarisationsfilter hinter dem PBSC aufgestellt und auf 45° zur Vertikalen gestellt, damit von beiden Strahlen nur der entsprechende Teil durchgelassen wird und diese durchgelassenen Komponenten interferieren können. Dahinter wird eine Photodiode aufgestellt, deren Photostrom in eine dazu proportionale Spannung umgewandelt wird. Diese wird mit einem angeschlossenen Oszilloskop gemessen. Nun wird der Glasplattenhalter in den Strahlengang im Interferometer positioniert, sodass je eine der Glasplatten in einem der gegenläufigen Teilstrahlen steht.

Die Messreihe zur Bestimmung des Kontrastes des Interferometers in Abhängigkeit vom eingestellten Winkel des ersten Polarisationsfilters wird wie folgt durchgeführt: Die Glasplattengeometrie wird bei fester Einstellung des Polarisationsfilters kontinuierlich so gedreht, dass auf dem Oszilloskop erkannt werden kann, zwischen welchen zu den Intensitäten proportionalen Spannungen sich das Signal bewegt. Diese Spannungen werden mit dem Cursor abgelesen und die nächste Einstellung des Polarisationsfilters kann überprüft werden.

\subsection{Messung des Brechungsindexes von Glas}
\label{subsec:glas}
Um den Brechungsindex von Glas zu bestimmen, werden ein zweiter, schräg stehender PBSC mit Spiegeln zum Interferieren und räumlichen Trennen der aus dem Interferometer kommenden Teilstrahlen, sowie zwei Photodioden benutzt. Mit dem Photostrom der Photodioden wird wieder jeweils eine dazu proportionale Spannung erzeugt. Diese Spannungen werden elektronisch voneinander subtrahiert. Da die Intensitäten der aus dem zweiten PBSC kommenden Strahlen zueinander komplementär sind, ergibt die Betrachtung der Spannungsdifferenz eine Methode zum Feststellen von $2\pi$-Phasenshifts zwischen den beiden Interferometerstrahlen anhand von Nulldurchgängen der Spannungsdifferenz. Die Anzhal der Nulldurchänge kann mit einer digitalen Zählvorrichtung gezählt werden. Da diese fehleranfällig ist, kann es hier sinnvoll sein, zusätzlich auch noch analog am Oszilloskop die Nulldurchänge zu zählen. Die Messung wird durchgeführt, indem die Glasplattengeometrie um 10 bis 14 Grad um die vertikale Achse gedreht wird und die Anzahl der $2\pi$-Phasenshifts abgelesen wird.


\subsection{Bestimmung des Brechungsindexes von Luft}
Zur Bestimmung des Brechungsindexes von Luft, wird anstelle der Glasplattengeometrie eine evakuierbare Zelle eingebaut, die nur von einem der gegenläufigen Interferometerstrahlen durchquert wird. Die Gaskammer wird evakuiert und anschließend langsam wieder mit Luft gefüllt, bis der Umgebungsdruck in der Zelle wieder erreicht ist. Durch diesen Vorgang ändert sich der Brechungsindex der enthaltenen Luft. Die Messreihe besteht auch hier im Zählen von $2\pi$-Phasenverschiebungen, wie schon im Abschnitt \ref{subsec:glas} beschrieben.
