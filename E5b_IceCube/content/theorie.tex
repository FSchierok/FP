\section{Theorie}
\label{sec:Theorie}
Das IceCube Experiment ist ein Array aus Photodedektoren, die am gepgrafischen Südpol im Eis eingefroren sind. Dabei befinden sich 5160 Photodedektoren
in einer Tiefe zwischen $\SI{1450}{\meter}$ und $\SI{2450}{\meter}$. Das Ziel ist Neutrinos zu detekieren, dazu wird das Cerenkovlicht der sekundär Teilchen
aus Neutrinowechselwirkungen gesucht. Da aber noch viele andere Teilchen das IceCube durchqueren, müssen die Daten gefiltert werden. Ein erster filter ist die
Richtung, aus der das Teilchen kam. Die gesuchten Neutrinos sollen von Untern kommen, so das die Erde als Abschirmung gegen die restliche Kosmische Strahlung
wirkt. Da die Richtungsauflösung aber nicht so genau ist, müssen weitere Filter angewand werden, um ein signifikantes Signal zu erhalten. Dazu werden mit Hilfe
von machine learning Algorythmen trainiert, die die weitere Auswertung durchführen.
\subsection{machine learning}
Machine learning wird oft zur Klassifizierung von Daten eingesetzt. Ein Algorythmus muss dazu mit bereits klasifizierten Daten trainiert werden. Häufig hat ein
Datensatz eine vielzahl an Attributen oder Features. Es können alle Features an der Lerner übergeben werden, dies erhöht aber die Rechenzeit. 
