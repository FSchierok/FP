\section{Diskussion}
\label{sec:Diskussion}

Die horizontale Komponente des Erdmagnetfelds wurde auf $B_\text{Erde,\,hor} = \SI{45.7\pm 0.6}{\micro \tesla}$ bestimmt. Laut dem Deutschen Geoforschungszentrum Potsdam \cite{mag} betrug die horizontale Erdmagnetfeldkomponente in Dortmund zum betrachteten Zeitpunkt $B_\text{Erde,\,hor\,,lit} = \SI{19.3}{\micro \tesla}$. Die Abweichung hat höchstwahrscheinlich einen systematischen Ursprung, da die restlichen Messungen nah an den Literaturwerten liegen. Verträglich mit dem Erfolg der folgenden Messungen, wäre beispielsweise ein nicht gewünschter, zeitlich konstanter Strom auf einer der beiden Horizontalfeldspulen, der durch eine falsche Eichung der beiden zugehörigen Stromregler entstanden sein könnte. Der Effekt wäre eine nicht näher bekannte vertikale Verschiebung der in Abbildung \ref{fig:plot1} dargestellten Daten, was nur die Ordinatenachsenabschnitte der Regressionsgeraden, nicht aber deren Steigungen verändern würde. Dies könnte am einfachsten überprüft und korrigiert werden, indem nicht die Anzahl der Umdrehungen des Reglers in einen Strom umgerechnet wird, sondern der Strom direkt gemessen wird, z.B. mit einem einfachen Multimeter.

Die gemessenen Landé-Faktoren lauten

\begin{eqnarray}
	g_{F,\, \text{Rb}87} = \SI{0.505 \pm 0.005},\\
	g_{F,\, \text{Rb}85} = \SI{0.337 \pm 0.002}.
\end{eqnarray}

Um die von der Literatur \cite{Rb} angegebenen Wert für die Kernspins zu erhalten müsste gelten:

\begin{eqnarray}
g_{F,\, \text{Rb}87, \, \text{lit}} = \SI{0.500},\\
g_{F,\, \text{Rb}85, \, \text{lit}} = \SI{0.333}.
\end{eqnarray}

Die Abweichungen zwischen unserer Messung und dem Literaturwert sind innerhalb einer bzw. zwei Standardabweichungen und im einstelligen Prozentbereich, was für eine gute Qualität der Messung spricht. Die noch vorhandenen Abweichungen könnten weiter verringert werden, wenn nicht nur eine schwarze Decke zur Verdunkelung verwendet würde sondern beispielsweise ein eigens dafür abgedunkelter Raum und die Magnetfelder nicht durch die Relation für das Magnetfeld innerhalb von Helmholtzspulen \eqref{helm} berechnet würden, sondern direkt mithilfe einer zentral positionierten Hall-Sonde vermessen würden. Vollständig vernachlässigt werden kann jedoch der quadratische Zeeman-Effekt, da die verwendeten Magnetfelder nicht groß genug sind um relevant zu sein. 

Die gemessenen und der Literatur entnommenen \cite{Rb} Werte für die Kernspins der beiden Isotope lauten:

\begin{eqnarray}
I_{\text{Rb}87} =& \SI{1.48 \pm 0.02}; \quad &I_{\text{Rb}85} = \SI{2.47 \pm 0.02},\\
I_{\text{Rb}87, \,\text{lit}} =& \SI{1.50}; \quad &I_{\text{Rb}85, \,\text{lit}} = \SI{2.50}.
\end{eqnarray}

Hier gelten die gleichen Anmerkungen wie zu den ermittelten Landé-Faktoren bezüglich der Qualität der Messung und möglichen Verbesserungen des Aufbaus.

Die bestimmten Anteile der Isotopen von $P_{\text{Rb}87} = (32.5 \pm 0.3)\%$ und $P_{\text{Rb}85} = (67.5 \pm 0.3) \%$ liegen einige Standardabweichungen entfernt von den in der Literatur \cite{Rb} genannten Daten von $P_{\text{Rb}87, \, \text{lit}} = 27,8\%$ und $P_{\text{Rb}85, \, \text{lit}} = 72,2 \%$. Die größeren Abweichungen lassen sich mit der ungenauen Messmethode erklären. Für eine genaue Messung sollte hier lieber auf das Zählen von Pixeln im Screenshot des Oszilloskops verzichtet werden. Für eine schnelle Probe der auftretenden Größenordnungen ist die Messung jedoch sehr gut geeignet.