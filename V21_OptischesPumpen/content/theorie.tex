\section{Theorie}
\label{sec:Theorie}
In diesem Experiment soll die Energiedifferenz der Zeemannniveaus untersucht werden. Dazu wird das Verfahren des Optischen Pumpen verwendet, was eine präzise Bestimmung ermöglicht.

Die Elektronen in einem Atom können nur diskrete Energien annehmen. Die Besetzungen $N_1$ und $N_2$  zweier Energielevel $W_1$ und $W_2$ (mit $W_1<W_2$) wird duch die Boltzmannverteilung beschrieben.
\begin{equation}
	\frac{N_2}{N_1}=\frac{g_2}{g_1}\frac{\exp(-W_2k_\mathrm{B}T)}{\exp(-W_1k_\mathrm{B}T)}
	\label{eqn:Bolztmann}
\end{equation}
Dabei ist $k_\mathrm{B}$ die Bolztmannkonstante, $T$ die Temperatur und $g_i$ die statistischen Gewichte. Das Optische Pumpen ermöglicht es diese natürliche Verteilung so anzuregen, das $N_2 > N_1$ gilt. Mit einem induzierten Strahlenübergang kann dann die Energiedifferenz $W_2-W_1$ genau bestimmt werden.
\subsection{Magnetische Momente}
Da sowohl der Bahndrehimpuls $\vec{L}$ als auch der Spin $\vec{S}$ ein magnetisches Moment $\vec{\mu_L}$ bzw. $\vec{\mu_S}$ haben, hat auch die Kopplung $\vec{J}$ ein magnetisches Moment $\vec{\mu_J}$.
\begin{align}
	\vec{\mu_J}&=\vec{\mu_L} +\vec{\mu_S} & \left| \mu_J \right|&=g_J\mu_B\sqrt{J(J+1)}
\end{align}
Dabei präzediert $\vec{\mu_J}$ um $\vec{J}$, und daher verschwinden über die Zeit gemittelt die senkrechten Anteile.
Wenn ein äußeres Magnetfeld $B$ anliegt, präzediert $\vec{\mu_J}$ auch um die parallelen Anteile von $B$. Die so beeinflussten Energieniveaus werden mit der Quantenzahl $M_J$ beschrieben und folgen der Formel
\begin{equation}
	U_\mathrm{mag} = M_J g_J \mu_\mathrm{B}B
\end{equation}
mit
\begin{equation}
	g_J=\frac{3J(J+1)+S(S+1)-L(L+1)}{2J(J+1)} \quad \text{und}\quad M_J \in [-J,...,J]
\end{equation}
Dies ist der Zeemann-Effekt.
\subsection{Kernspin}
Ein weitere Einfluss kommt aus dem Kernspin $\vec{I}$. Er koppelt ebenfalls an $\vec{J}$ und erzeugt die Hyperfeinstruktur.
\begin{equation}
	\vec{F}=\vec{J} +\vec{I}
\end{equation}
