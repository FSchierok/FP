\section{Theorie}
\label{sec:Theorie}
In diesem Experiment soll die Energiedifferenz der Zeemannniveaus untersucht werden. Dazu wird das Verfahren des Optischen Pumpen verwendet, was eine präzise Bestimmung ermöglicht.

Die Elektronen in einem Atom können nur diskrete Energien annehmen. Die Besetzungen $N_1$ und $N_2$  zweier Energielevel $W_1$ und $W_2$ (mit $W_1<W_2$) wird duch die Boltzmannverteilung beschrieben.
\begin{equation}
	\frac{N_2}{N_1}=\frac{g_2}{g_1}\frac{\exp(-W_2k_\mathrm{B}T)}{\exp(-W_1k_\mathrm{B}T)}
	\label{eqn:Bolztmann}
\end{equation}
Dabei ist $k_\mathrm{B}$ die Bolztmannkonstante, $T$ die Temperatur und $g_i$ die statistischen Gewichte. Das Optische Pumpen ermöglicht es diese natürliche Verteilung so anzuregen, das $N_2 > N_1$ gilt. Mit einem induzierten Strahlenübergang kann dann die Energiedifferenz $W_2-W_1$ genau bestimmt werden.
\subsection{Magnetische Momente}
Da sowohl der Bahndrehimpuls $\vec{L}$ als auch der Spin $\vec{S}$ ein magnetisches Moment $\vec{\mu_L}$ bzw. $\vec{\mu_S}$ haben, hat auch die Kopplung $\vec{J}$ ein magnetisches Moment $\vec{\mu_J}$.
\begin{align}
	\vec{\mu_J}&=\vec{\mu_L} +\vec{\mu_S} & \left| \mu_J \right|&=g_J\mu_B\sqrt{J(J+1)}
\end{align}
Dabei präzediert $\vec{\mu_J}$ um $\vec{J}$, und daher verschwinden über die Zeit gemittelt die senkrechten Anteile.
Wenn ein äußeres Magnetfeld $B$ anliegt, präzediert $\vec{\mu_J}$ auch um die parallelen Anteile von $B$. Die so beeinflussten Energieniveaus werden mit der Quantenzahl $M_J$ beschrieben und folgen der Formel
\begin{equation}
	U_\mathrm{mag} = M_J g_J \mu_\mathrm{B}B
\end{equation}
mit
\begin{equation}
	g_J=\frac{3J(J+1)+S(S+1)-L(L+1)}{2J(J+1)} \quad \text{und}\quad M_J \in [-J,...,J]
\end{equation}
Dies ist der Zeemann-Effekt.
\subsection{Kernspin}
Ein weitere Einfluss kommt aus dem Kernspin $\vec{I}$. Er koppelt ebenfalls an $\vec{J}$ und erzeugt dadurch die Hyperfeinstruktur.
\begin{equation}
	\vec{F}=\vec{J} +\vec{I}
\end{equation}
$F$ läuft zwischen $J+I$ und $|J-I|$ und kann dementsprechend $2J+1$ bzw. $2I+1$ Werte annehmen. Die dadurch entstehenden Energieniveaus sind gegeben durch:
\begin{equation}
	U_\mathrm{HF}=g_F\mu_BB
	\label{eqn:zeemann}
\end{equation}
mit
\begin{equation}
	g_F=g_J\frac{F(F+1)+J(J+1)-I(I+1}{2F(F+1)}
	\label{eqn:g_F}
\end{equation}
\subsection{Optisches Pumpen}
Im Folgenden wird ein Alkali-Atom ohne Drehimpuls angenommen. Es besitz somit einen Grundzustand ${}^2\mathrm{S}_\dfrac{1}{2}$ und die zwei ersten angeregten Zustände ${}^2\mathrm{P}_\dfrac{1}{2}$ und ${}^2\mathrm{P}_\dfrac{3}{2}$. Die möglichen Übergänge ergeben das $\mathrm{D}_1-\mathrm{D}_2$ Dublett.
In ${}^2\mathrm{S}_\dfrac{1}{2}$ und ${}^2\mathrm{P}_\dfrac{1}{2}$ ist $J=\dfrac{1}{2}$ und daher $M_J auf \pm \dfrac{1}{2}$ beschrängt. Daraus können drei $\Delta M$ resultieren.
\begin{description}
	\item[$\Delta M=1$] Das Licht ist rechtszirkular polarisiert ($\sigma^{+}$)
	\item[$\Delta M=-1$] Das Licht ist linkszirkular polarisiert ($\sigma^{-}$)
	\item[$\Delta M=0$] Das Licht ist linear polarisiert (\pi)
\end{description}
Wenn nun mit rechtszirkular polarisiertes Licht angeregt wird, also $\Delta M=1$ gelten muss, ist nur der Übergang ${}^2\mathrm{S}_\dfrac{1}{2}, M_J=-\dfrac{1}{2}$ zu ${}^2\mathrm{P}_\dfrac{1}{2}, M_J=\dfrac{1}{2}$ möglich. Durch spontane Emission fallen die angeregten Elektronen dann zufällig auf die tieferen Niveaus zurück.
Durch Wiederholung dieses Vorgangs wird ${}^2\mathrm{S}_\dfrac{1}{2}, M_J=-\dfrac{1}{2}$ also immer leerer und ${}^2\mathrm{S}_\dfrac{1}{2},M_J=\dfrac{1}{2}$ immer voller, entgegen der natürlichen Verteilung. Wenn keine Elektronen aus ${}^2\mathrm{S}_\dfrac{1}{2}, M_J=-\dfrac{1}{2}$ angeregt werden kann, wird kein Licht absorbiert, bzw die Transperenz nimmt zu.
\subsection{Messung der Zeemannniveaus}
Die angeregten Elektronen können über zwei Mechanismen in einen niedrigeren Zustand über gehen. Entweder über die spontane Emission oder über die induzierte Emission. Bei der letzteren kann ein Photon, das genau die Energiedifferenz der Energieniveaus besitz eben diesen Übergang anstoßen. Bei den in diesem Experiment untersuchten Energiedifferenzen ist die induzierte Emission deutlich überwiegend.
\subsection{Quadratischer Zeemann-Effekt}
Für größere Magnetfelder muss ein weitere Term in die Formel \eqref{eqn:zeemann} mit aufgenommen werden. Dadurch ergibt sich:
\begin{equation}
	U_\mathrm{HF}=g_F\mu_BB + g_F^2\mu_B^2B^2\frac{1-2M_F}{\Delta E_\mathrm{Hy}}
\end{equation}
mit der Hyperfeinstrukturaufspaltung $\Delta E_{Hy}$. Es fällt auch auf, dass $U_\mathrm{HF}$ jetzt von $M_F$ abhängt.
