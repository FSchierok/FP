\section{Theorie}
\label{sec:Theorie}
\subsection{Vakuum}
Ein Vakuum ist ein Volumen, das frei von Materie ist. Im Umgangssprachlichen ist damit oft Luftleer gemeint, in wissenschaftlichen Sinne sind aber auch andere
Gase oder Partikel gemeint. Da es nicht möglich ist ein perfektes Vakuum zu erstellen, wird zwischen verschiedenen Güteklassen unterschieden. Diese
Unterscheiden ich in ihrem Unterdruck, also dem Druckunterschied zum Normaldruck.
\begin{description}
	\item[Normaldruck] $\SI{101.325}{\kilo \pascal}$
	\item[Grobvakuum] $\SI{30}{\kilo \pascal}$
	\item[Feinvakuum] $\SI{100}{\pascal}$ bis $\SI{0.1}{\pascal}$
	\item[Hochvakuum (HV)] $\SI{e-1}{\pascal}$ bis $\SI{e-5}{\pascal}$
	\item[Ultrahochvakuum (UHV)] $\SI{e-5}{\pascal}$ bis $\SI{e-10}{\pascal}$
	\item[extrem hohes Vakuum (XHV)] < $\SI{e-10}{\pascal}$
	\item[ideales Vakuum (IV)] $\SI{0}{\pascal}$
\end{description}
\subsection{Gasphysik}
Die einfachste Besceschreibung für Gase ist die ideale Gas Gleichung:
\begin{equation}
	pV=Nk_\text{B}T
\end{equation}
Diese gilt, wenn das Gas aus identischen Punktmassen besteht, diese nur über elastische Stöße wecheslwirken und das Gas in einem Volumen $V$ eingeschlossen
ist, mit dessen Begrenzung auch nur elastich gestoßen wird. Aus dieser Formel ist auch nochmal ersichtlich, dass $p=0$ nicht möglich ist, da dann $T=0$ gelten
müsste, was nach dem 3. Hauptsatz der Thermodynamik Unerreichbar ist. Es lässt sich auch das Gesetz von Boyle-Mariotte herleiten, das besagt, dass bei
 konstanter Temperatur $p$ antiproportional zu $V$ ist.
\subsection{Druck}
Der Druck gibt die Kraft an, die ein Gas auf sein Behältnis ausübt. Er wird in $\si{\pascal}=\si{\newton \per \meter \squared}$ gemessen. Falls ein Gasgemisch
 vorliegt, steht jeder Gasanteil unter einem partial Druck $p_i$.
