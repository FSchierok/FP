\section{Theorie}
\label{sec:Theorie}
Im folgenden werden die Grundlagen der Vakuumtechnik erklärt, sowie einige Begriffe definiert. Diese Informationen stammen, falls nicht explizit anders gesagt, von Pfeiffer Vacuum\cite{Pfeiffer}.
\subsection{Vakuum}
Ein Vakuum ist ein Volumen, das frei von Materie ist. Im Umgangssprachlichen ist damit oft Luftleer gemeint, in wissenschaftlichen Sinne sind aber auch andere
Gase oder Partikel gemeint. Da es nicht möglich ist ein perfektes Vakuum zu erstellen, wird zwischen verschiedenen Güteklassen unterschieden. Diese
Unterscheiden ich in ihrem Unterdruck, also dem Druckunterschied zum Normaldruck.
\begin{description}
	\item[Normaldruck] $\SI{101.325}{\kilo \pascal}$
	\item[Grobvakuum] $\SI{30}{\kilo \pascal}$
	\item[Feinvakuum] $\SI{100}{\pascal}$ bis $\SI{0.1}{\pascal}$
	\item[Hochvakuum (HV)] $\SI{e-1}{\pascal}$ bis $\SI{e-5}{\pascal}$
	\item[Ultrahochvakuum (UHV)] $\SI{e-5}{\pascal}$ bis $\SI{e-10}{\pascal}$
	\item[extrem hohes Vakuum (XHV)] < $\SI{e-10}{\pascal}$
	\item[ideales Vakuum (IV)] $\SI{0}{\pascal}$
\end{description}
\subsection{Gasphysik}
Die einfachste Besceschreibung für Gase ist die ideale Gas Gleichung:
\begin{equation}
	pV=Nk_\text{B}T
\end{equation}
Diese gilt, wenn das Gas aus identischen Punktmassen besteht, diese nur über elastische Stöße wecheslwirken und das Gas in einem Volumen $V$ eingeschlossen
ist, mit dessen Begrenzung auch nur elastich gestoßen wird. Aus dieser Formel ist auch nochmal ersichtlich, dass $p=0$ nicht möglich ist, da dann $T=0$ gelten
müsste, was nach dem 3. Hauptsatz der Thermodynamik Unerreichbar ist. Es lässt sich auch das Gesetz von Boyle-Mariotte herleiten, das besagt, dass bei
 konstanter Temperatur $p$ antiproportional zu $V$ ist. Zudem kann zu der Teilchendichte umgestellt werden.
 \begin{equation}
 	\frac{N}{V}=\frac{k_\text{B}T}{p}
 \end{equation}
 \cite{TuSKierfeld}
\subsection{Druck}
Der Druck gibt die Kraft an, die ein Gas auf sein Behältnis ausübt. Er wird in $\si{\pascal}=\si{\newton \per \meter \squared}$ gemessen. Falls ein Gasgemisch
vorliegt, steht jeder Gasanteil unter einem partial Druck $p_i$.
\subsection{Mittlere freie Weglänge}
Die mittlere freie Weglänge gibt an, wie weit ein Teilchen sich statistisch bewegen kann, bis es mit einem anderen Teilchen wecheslwirkt.
Sie wird von der Formel
\begin{equation}
	l=\frac{k_\text{B}T}{\sqrt{2}\pi p d^2}
\end{equation}
beschrieben. Hierbei ist $d$ der Moleküldurchmesser. Es fällt auf, dass $l$ Temperaturabhänig ist.

\subsection{Drehschieberpumpe}
Eine Drehschiberpumpe besteht aus einem Zylindrischen Hohlraum, in dem exzentrisch ein Rotor gelagert ist. An diesem Rotor sind zwei oder drei Drehschieber,
die durch eine Feder den variablen Abstand zur Außenwand abschließen. Der Pumpeneinzug ist mit dem großen Volumen verbunden, der Ausstoß mit dem Kleinen.
Wenn Rotor nun gedreht wird, strömt Gas in die große Kammer ein. Dieses Gas wird dann von dem nachfolgenden Drehschieber isoliert. Das eingeschlossene Gas wird
durch die Drehung immer weiter komprimiert, bis der Überdruck ausreicht um das Auslassventil zu öffnen.
\subsection{Turbomolekularpumpe}
Eine Turbomolekularpumpe besteht aus einem Hohlzylinder, in dem abwechselnd Rotorn und Stratorn sind. Dabei die Rotorblätter in Drehrichtung gekippt und die
Stratorblätter entgegen. Wenn nun ein Molekül einen Rotor trifft, wird es in Drehrichtung beschleunigt. Das darauf folgende Stratorblatt lenkt das Molekül dann
in den nächsten Rotor. Es ist für die Funktionsweise wichtig, dass Geschwindigkeit der Rotorblätter größer ist, als die mittlere Molekulargeschwindigkeit das
Gases. Zudem müssen die Abstände der Rotoren in der Größenordnung der mittleren freien Weglänge der Gasmoleküüle liegen, da die Bewegung sonst zum erliegent
kommt.
\subsection{Vakuummeter}
\subsubsection{Kalt-/Glühkathoden}
Bei einem Vakuummeter wird zwischen einer Kathode und einer Anode eine Spannung von ca $\SI{2}{\kilo \volt}$ angelegt. Dadurch werden Elektronen
beschleunigt, die auf ihrem Weg mit Gasteilchen stoßen und diese ionisieren. Diese geladenen Teilchen werden zu der Kathode beschleunigt und es lässt sich ein
Strom messen. Dieser ist abhänig von dem Druck bzw. der Anzahl an Stößen, die ausgeführt werden. Es wird unterschieden zwischen Kaltkathoden, bei den bereits
vorhandene Elektronen beschleunigt werden, und Glükathoden, bei denen mit dem Glühelerktrischen-Effekt Elektronen aus der Kathode ausgelöst werden.
\subsubsection{??}
Beim ??-Vakuummeter wird die Abhänigkeit der Wärmeleitfähigkeit vom Druck ausgenutzt. Dazu wird ein Draht elektrisch erwärmt. Dieser gibt ein Teil der
Wärmeenergie an das umgebene Gas ab, der Rest staut sich im Draht auf. Da bei niedriegeren Druck die Wärmeleitung abnimmt, wird der Draht umso wärmer, je
geringer der Umgebungsdruck ist. Die Temperatur des Drahtes kann über sein elektrischen Wiederstande gemessen werden.
\subsection{Lecks}
Ein Vakuum kann auf zwei Arten Lecks haben, reale und virtuelle. Bei realen Lecks kann Gas von außerhalb des Aufbaus in den Rezipenten eindringen, zB. durch
nicht richtig abgedichtete Flansche. Diese Lecks kann man einfach Abdichten, wenn man sie lokalisiert hat. Bei virtuellen Lecks strömten Teilchen aus dem
Aufbau selber in den Rezipenten, zB. Einschlüsse in Metallteilen. Diese lassen sich in den meisten Fällen nicht abdichten, sondern müssen bei der Produktion
beseitigt werden.
\subsection{Generelle Definitionen}
\begin{description}
	\item[Absorbtion] Aufnahme eines Teilchens in einen Festkörper
	\item[Adsorbtion] Ablagerung eines Teilchens auf der Oberfläche eines Festkörpers
	\item[Desorbtion] Ablösen eines Teilchens von der Oberfläche eines Festkörpers
	\item[Diffusion] Vermischung von Gasen bzw. Flüssigkeiten durch Molekularbewegung
	\item[Laminare Stömung] Wirbelfreie Strömung
	\item[Saugleistung] $\dot{Q}$, Stoffmenge pro Zeit
\end{description}
\subsection{Leitwert}
Nur sehr selten wird eine Pumpe direkt an den Versuch angeschlossen, in der Regel wird ein Schlauchsystem eingesetzt. Dieses hat neben der Volumensvergrößerung
noch mehr negative Auswirkungen auf das reale Saugvermögen der Pumpe. Es entsteht ein Druckgefälle zur Pumpe hin, bedingt durch den Strömungswiederstand des Schlauchsystems. In der Regel wird der Kehrwert dieses Wiederstandes benutzt, der Leitwert. Der Leitwert verhält sich analog zu $\dfrac{1}{R}$ in elektrischen
Systemen:
\begin{align}
	L_\text{ges,parallel}=&\sum_i L_i \\
	L_\text{ges,reihe}=&\sum_i \frac{1}{L_i}
\end{align}
Die Abhänigkeit des Leitwertes vom Druck lässt sich in drei Bereiche aufteilen, Abhänig von der Strömungsart. Bei viskoer Strömung ist der Leitwert linear zum
Druck und bei molekularer Strömung konstant. Der Übergangsbereich lässt sich mit der Summe des laminaren und des molekularen Leitwertes annähern.
Es gilt für Raumtemperatur:
\begin{align}
	L_\text{Zylinder,laminar}=& \num{1.35}\frac{d^4}{l}\overline{p} \\
	L_\text{Zylinder, molekular}=& \num{11.6}\frac{\pi d^3}{3l}
\end{align}
Dabei ist $l$ die Rohrlänge und $d$ der Rohrdurchmesser und $\overline{p}$ der mittlere Druck im Rohr.
