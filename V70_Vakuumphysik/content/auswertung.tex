\section{Auswertung}

\subsection{Fehlerrechnung}

Die Fortpflanzung von Messungenauigkeiten für mehrere unabhängige Fehler wird durch die Gaußsche Fehlerfortpflanzung

\begin{equation}
\Delta f = \sqrt{\sum_{i \, = \, 1}^{n} \, \left(\frac{\partial f}{\partial x_i} \, \Delta x_i\right)^2}
\label{fehler}
\end{equation}

beschrieben. Dabei gibt $\Delta x$ die Unsicherheit des arithmetischen Mittelwerts $\bar{x}$ einer Observablen $x$ an:

\begin{equation}
\Delta x = \frac{1}{\sqrt{n}} \sqrt{\frac{1}{n-1} \sum_{i \, = \, 1}^{n} \, \left(\bar{x}- x_i\right)^2}.
\end{equation}

Die Zahl $n$ gibt die Anzahl der unabhängigen Messungen an.


\subsection{Bestimmung des Volumens des Versuchaufbaus}

Es wird angenommen, dass alle verwendeten Komponenten zylindrisch sind. Das Volumen eines Zylinders berechnet sich durch

\begin{equation}
V_\text{zyl} = \frac{\pi d^2 z}{4},
\end{equation}

wobei $d$ den Durchmesser des Zylinders und $z$ dessen Höhe bezeichnet. Mit der Fehlerfortpflanzung \eqref{fehler} folgt:

\begin{equation}
\Delta V_\text{zyl} = \frac{\pi}{4} \sqrt{(2 d z \Delta d)^2 + (d^2 \Delta z)^2}.
\end{equation}

Die gemessenen Durchmesser $d$ und Höhen $z$ sowie die abgeschätzten, zugehörigen Unsicherheiten $\Delta d$ und $\Delta z$ der verwendeten Komponenten lassen sich der Tabelle \ref{tab1} (im Anhang) entnehmen. Die Angabe über das Volumen des Tanks wurden der Versuchsanleitung \cite{V70} entnommen; die Maße der Turbomolekularpumpe fanden sich in dem beim Versuch ausliegenden Bauplänen des Pumpen-Herstellers. In der letzten Spalte wird angegeben, ob und wie oft das beschriebene Bauteil bei der Vermessung der Drehschieberpumpe (\enquote{D}) bzw. der Turbomolekularpumpe (\enquote{T}) verwendet wurde. Die Gesamtvolumina für beide Versuchsteile ergeben sich durch Addition der Teilvolumina, wobei die Ungenauigkeit einer Summe von fehlerbehafteten Volumina $V_1$ und $V_2$ gegeben ist durch $\Delta (V_1 + V_2) = \sqrt{\left(\Delta V_1^2 + \Delta V_2^2\right)}$:

\begin{align}
	V_\text{Dreh} &= \SI{11.2+-0.8}{\litre}\\
	V_\text{Turbo} &= \SI{11.0+-0.8}{\litre}
\end{align}


\subsection{Messung der Sauggeschwindigkeit der Drehschieberpumpe}

\subsubsection{Aufnahme der Evakuierungskurve der Drehschieberpumpe}

Die Messwerte des Drucks $p(t)$ in Abhängigkeit von der Zeit sind in Tabelle \ref{tab:tab2} (im Anhang) aufgelistet. Als Messungenauigkeit des Drucks wird ein relativer Fehler von 10\% angesetzt. Der Umgebungsdruck wird nach dem Wetterbericht des Tages der Versuchsdurchführung \cite{wetter} auf $p_0 = \SI{1017 +-3}{\milli \bar}$ abgeschätzt. Als minimaler, mit der Pumpe erreichbarer Druck wurde $p_\text{E} = \SI{0.05+-0.005}{\milli \bar}$ gemessen. Es wird ein exponentieller Abfall des Drucks erwartet. Eine Ausgleichsrechnung für die logarithmierten Messwerte

\begin{align}
	f &= a \cdot t + b \quad \text{mit} \quad f = \ln \left(\frac{p(t) -p_\text{E}}{p_0 - p_\text{E}}\right)\\
	\Delta f &= \sqrt{\left(\frac{\Delta p(t)}{p-p_\text{E}}\right)^2 + \left(\frac{\Delta p_0}{p_0 - p_\text{E}}\right)^2 + \left(\frac{p- p_0}{(p(t) - p_\text{E}) (p_0 - p_\text{E})} \Delta p_\text{E}\right)^2}
\end{align}

ergibt die Sauggeschwindigkeit

\begin{align}
	S &= - a \cdot V_\text{Dreh}\\
	\Delta S &= \sqrt{\left(a \Delta V_\text{Dreh}\right)^2 + \left(V_\text{Dreh} \Delta a\right)^2}.
\end{align}

Es stellt sich heraus, dass drei in guter Näherung lineare Abschnitte der logarithmierten Evakuierungskurve gefunden werden können (s. Abbildung \ref{fig:drehschieber}). Die für diese Bereiche berechneten Sauggeschwindigkeiten lassen sich der Tabelle \ref{tab3} entnehmen.

\begin{table}
	\begin{center}
		\begin{tabular}{cccccc}
			\toprule
			Bereich & $p \, / \, \text{mbar}$ & $a \, / \, \text{s}^{-1}$ &        $b$        & $S \, / \, (\text{l}/\text{s})$ &  \\ \midrule
			 Blau   &       1000 ... 20       &    $-0,069 \pm 0,007 $    & $ -1,04 \pm 0,2 $ &      $ 0,78 \pm 0,09     $      &  \\
			Orange  &       10 ... 0,6        &    $-0,096 \pm 0,002$     &  $0,01 \pm 0,1 $  &      $  1,08 \pm 0,08   $       &  \\
			 Grün   &      0,4 ... 0,06       &    $-0,057 \pm 0,003$     & $ -3,07 \pm 0,3 $ &      $  0,64 \pm 0,05   $       &  \\ \bottomrule
		\end{tabular}
		\caption{Die Parameter der Regressionsgeraden in Abbildung \ref{fig:drehschieber} für die Sauggeschwindigkeit der Drehschieberpumpe.}
		\label{tab3}
	\end{center}
\end{table}

\begin{figure}
	\centering
	\includegraphics[width=\linewidth]{img/drehschieber}
	\caption{Evakuierungskurve der Drehschieberpumpe.}
	\label{fig:drehschieber}
\end{figure}


\subsubsection{Leckratenmessung der Drehschieberpumpe}

Die aufgenommenen Daten der Leckratenmessung lassen sich der Tabelle \ref{tab5} (im Anhang) entnehmen. Eine lineare Regression mit

\begin{equation}
p(t) = a \cdot t + b
\end{equation}

bei den Gleichgewichtsdrücken $p_\text{Gg} = \SI{0.1}{\milli \bar}$, $p_\text{Gg} = \SI{0.4}{\milli \bar}$, $p_\text{Gg} = \SI{0.6}{\milli \bar}$ und $p_\text{Gg} = \SI{1}{\milli \bar}$ (jeweils mit einer abgeschätzten, relativen Messungenauigkeit von 10\%) ergibt über den Zusammenhang für die Leckraten


\begin{align}
	Q &= V_\text{Dreh} \cdot a\\
	\Delta Q &= \sqrt{\left(a \Delta V_\text{Dreh} \right)^2 + \left(V_\text{Dreh} \Delta a\right)^2}
\end{align}

die Sauggeschwindigkeiten

\begin{align}
	S &= \frac{Q}{p_\text{Gg}}\\
	\Delta S &= \sqrt{\left(\frac{\Delta Q}{p_\text{Gg}}\right)^2 + \left(\frac{Q \Delta p_\text{Gg}}{p_\text{Gg}^2}\right)^2},
\end{align}

welche in Tabelle \ref{tab4} aufgelistet sind. In Abbildung \ref{fig:drehLeck} sind die verwendeten Ausgleichsgeraden zu finden; in Abbildung \ref{fig:drehSaug} sind zusammenfassend für die Drehschieberpumpe alle ermittelten Sauggeschwindigkeiten in Abhängigkeit vom jeweils verwendeten Gleichgewichtsdruck dargestellt. 

\begin{table}
	\begin{center}
		\begin{tabular}{cccccccc}
			\toprule
			$p_\text{Gg} \, / \, \text{mbar}$ & $a \, / \, \text{s}^{-1}$ &       $b$        & $Q \, / \, (\text{mbar \cdot l}/\text{s})$ & $S \, / \, (\text{l}/\text{s})$ &  \\ \midrule
			0,1                &    $0,0050 \pm 0,0002$    & $0,15 \pm 0,02$  &          $ 0,056 \pm 0,005  $           &       $ 0,56 \pm 0,07  $        &  \\
			0,4                &     $0,035 \pm 0,001$     & $0,33 \pm 0,05$  &          $  0,39 \pm 0,03   $           &       $  1,0 \pm 0,1   $        &  \\
			0,6                &     $0,067 \pm 0,002$     & $0,55 \pm 0,07 $ &           $ 0,74 \pm 0,07   $           &        $1,2 \pm 0,2   $         &  \\
			1,0                &    $0,1187 \pm 0,0007$    & $0,96 \pm 0,03 $ &           $  1,3 \pm 0,1   $            &       $  1,3 \pm 0,2   $        &  \\ \bottomrule
			&                           &
		\end{tabular}
		\caption{Die Parameter der Regressionsgeraden in Abbildung \ref{fig:drehLeck} für die Leckratenmessung der Drehschieberpumpe.}
		\label{tab4}
	\end{center}
\end{table}


\begin{figure}
	\centering
	\includegraphics[width=\linewidth]{img/drehLeck}
	\caption{Regressionsgeraden für den Druck innerhalb der Apparatur während der Leckratenmessung mithilfe der Drehschieberpumpe.}
	\label{fig:drehLeck}
\end{figure}

\begin{figure}
	\centering
	\includegraphics[width=\linewidth]{img/DrehSaug}
	\caption{Zusammenfassender Plot für die Messwerte der Sauggeschwindigkeit der Drehschieberpumpe in Abhängigkeit vom Druck.}
	\label{fig:drehSaug}
\end{figure}


\subsection{Messung der Sauggeschwindigkeit der Turbomolekularpumpe}

\subsubsection{Aufnahme der Evakuierungskurve der Turbomolekularpumpe}

Analog zur Berechnung der Sauggeschwindigkeiten der Drehschieberpumpe folgen für Turbomolekularpumpe die in Tabelle \ref{tab8} aufgelisteten Sauggeschwindigkeiten. Zur Berechnung wurden die in Tabelle \ref{tab:tab7} (im Anhang) aufgelisteten Messwerte sowie $p_0 = \SI{5 +- 1}{\micro \bar}$ und $p_\text{E} = \SI{0.015 +- 0.005}{\micro \bar}$ verwendet. Es werden die beiden in Abbildung \ref{fig:turbo} dargestellten linearen Abschnitte unterschieden. 

\subsubsection{Leckratenmessung der Turbomolekularpumpe}

In Tabelle \ref{tab9} (im Anhang) finden sich die aufgenommenen Messdaten für diesen Versuchsabschnitt. Analog zur Leckratenmessung bei der Drehschieberpumpe ergeben sich für die eingestellten Gleichgewichtsdrücke $p_\text{Gg} = \SI{0.05}{\micro \bar}$, $p_\text{Gg} = \SI{0.1}{\micro \bar}$, $p_\text{Gg} = \SI{0.15}{\micro \bar}$ und $p_\text{Gg} = \SI{0.2}{\micro \bar}$ (auch jeweils mit 10 \% Messunsicherheit) die in Tabelle \ref{tab4} aufgelisteten Leckraten und Sauggeschwindigkeiten. Die verwendeten Regressionsgeraden sind der Abbildung \ref{fig:turboLeck} zu entnehmen; in Abbildung \ref{fig:turboSaug} befindet sich eine Übersicht über die gemessenen Sauggeschwindigkeiten in Abhängigkeit vom eingestellten Gleichgewichtsdruck.




	 
\begin{figure}
	\centering
	\includegraphics[width=\linewidth]{img/turbo}
	\caption{Evakuierungskurve der Turbomolekularpumpe.}
	\label{fig:turbo}
\end{figure}

\begin{table}
	\begin{center}
		\begin{tabular}{cccccc}
			\toprule
			Bereich & $p \, / \, \text{µbar}$ & $a \, / \, \text{s}^{-1}$ &       $b$        & $S \, / \, (\text{l}/\text{s})$ &  \\ \midrule
			 Blau   &       5 ... 0,06        &     $-0,79 \pm 0,05 $     & $ -0,3 \pm 0,1 $ &       $ 8,7 \pm 0,8     $       &  \\
			Orange  &      0,04 ... 0,02      &     $-0,09 \pm 0,02$      & $-4.8 \pm 0,3 $  &       $  1,0 \pm 0,2   $        &  \\ \bottomrule
		\end{tabular}
		\caption{Die Parameter der Regressionsgeraden in Abbildung \ref{fig:turbo} für die Sauggeschwindigkeit der Turbomolekularpumpe.}
		\label{tab8}
	\end{center}
\end{table}

\begin{table}
	\begin{center}
		\begin{tabular}{ccccc}
			\toprule
			$p_\text{Gg} \, / \, \text{µbar}$ & $a \, / \, \text{s}^{-1}$ &       $b$       & $Q \, / \, (\text{µbar \cdot l}/\text{s})$ & $S \, / \, (\text{l}/\text{s})$ \\ \midrule
			              0,05                &     $0,066 \pm 0,001$     & $0,00 \pm 0,03$ &             $ 0,73 \pm 0,05  $             &           $15 \pm 2 $           \\
			               0,1                &     $0,237 \pm 0,006$     & $-0,3 \pm 0,2$  &             $  2,6 \pm 0,2   $             &         $  26 \pm 3   $         \\
			              0,15                &     $0,356 \pm 0,009$     & $-0,2 \pm 0,2 $ &              $ 3,9 \pm 0,3 $               &          $26 \pm 3   $          \\
			               0,2                &      $0,48 \pm 0,01$      & $-0,1 \pm 0,1 $ &             $  5,3 \pm 0,4  $              &         $  27 \pm 3   $         \\ \bottomrule
		\end{tabular}
		\caption{Die Parameter der Regressionsgeraden in Abbildung \ref{fig:turboLeck} für die Leckratenmessung der Turbomolekularpumpe.}
		\label{tab4}
	\end{center}
\end{table}


\begin{figure}
	\centering
	\includegraphics[width=\linewidth]{img/turboLeck}
	\caption{Regressionsgeraden für den Druck innerhalb der Apparatur während der Leckratenmessung mithilfe der Turbomolekularpumpe.}
	\label{fig:turboLeck}
\end{figure}


\begin{figure}
	\centering
	\includegraphics[width=\linewidth]{img/TurboSaug}
	\caption{Zusammenfassender Plot für die Messwerte der Sauggeschwindigkeit der Turbomolekularpumpe in Abhängigkeit vom Druck.}
	\label{fig:turboSaug}
\end{figure}
