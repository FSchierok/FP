\section{Diskussion}
\label{sec:Diskussion}

Mithilfe der aufgenommenen Evakuierungskurven (Abb. \ref{fig:drehschieber} und \ref{fig:turbo}) der beiden Pumpen konnte festgestellt werden, dass die Sauggeschwindigkeit der Pumpen druckabhängig ist. Die Messergebnisse für die Sauggeschwindigkeit der Drehschieberpumpe, die mithilfe der Evakuierungskurve und der Leckratenmessung ermittelt wurden, decken sich im Rahmen der Messungenauigkeit. Die Herstellerangabe von $\SI{1.1}{\litre \per \second}$ (siehe \cite{V70}) konnte im Bereich von $0,6$ bis $\SI{10}{\milli \bar}$ durch die Vermessung der Evakuierungskurve bestätigt werden. Auch die Leckratenmessung liefert in diesem Bereich im Rahmen der Messungenauigkeit mit der Herstellerangabe kompatible Sauggeschwindigkeiten (siehe Abb. \ref{fig:drehSaug}). Für hohe Drücke ($20$ bis  $\SI{1000}{\milli \bar}$) und Drücke unterhalb von  $\SI{0.4}{\milli \bar}$ konnte die Drehschieberpumpe nicht die vom Hersteller angegebene Sauggeschwindigkeit erreichen.

Die Herstellerangabe der Sauggeschwindigkeit der Turbomolekularpumpe von $\SI{77}{l/s}$ (siehe ebenfalls \cite{V70}) konnte durch die Messungen nicht bestätigt werden. Die Messwerte aus der Leckratenmessugng  befinden sich deutlich unterhalb dieser Angabe und sind auch im Rahmen der Messungenauigkeit nicht mit denen aus der Evakuierungskurve gewonnenen Daten vereinbar (siehe Abb. \ref{fig:turboSaug}). 

Die beobachteten Abweichungen zwischen unseren Messungen und den Herstellerangaben lassen sich primär durch den Strömungswiderstand der Rohre erklären. Gerade für den Druckbereich, in dem die Turbomolekularpumpe arbeitet, ist der Durchmesser der Rohre zu klein um ihren Leitwert vernachlässigen zu können: Der effektive (gemessene) Strömungsgeschwindigkeit berechnet sich aus dem Leitwert $L$ und der theoretisch maximal möglichen Sauggeschwindigkeit $S$ (hier: der Herstellerangabe) zu

\begin{equation}
	S_\text{eff} = \frac{S}{1 + \frac{S}{L}}.
\end{equation}

Der Leitwert ist dabei nach dem Hagen-Poiseuille-Gesetz proportional zum Rohrdurchmesser in der vierten Potenz für laminare Strömungen bzw. proportional zu dessen dritter Potenz für molekulare Strömungen. Gerade sich verjüngende Rohrelemente im Versuchsaufbau führen daher zu Abweichungen. Des Weiteren ist zu bedenken, dass der genutzte Rezipient ein zu geringes Volumen für die genutzte Pumpleistung besitzt. Dies verursacht eine starke Wechselwirkung der Gasmoleküle mit der Oberfläche des Rezipienten. Eine zusätzliche Fehlerquelle liegt in den stets vorhandenen Desorptionseffekten. Genauere Messwerte hätten erzielt werden können, wenn mithilfe einer Schaltungselektronik die Ventile nicht per Hand und nicht von zwei verschiedenen Personen hätten geschlossen werden müssen. Auch eine genauere, automatisierte Zeit- und Druckmessung wäre für genauere Messwerte förderlich. Zusätzlich muss angemerkt werden, dass der Hersteller zum Testen der Turbomolekularpumpe Stickstoff anstelle von Luft (wie im durchgeführten Versuch) verwendet hat, wodurch sich die Sauggeschwindigkeit der Pumpe noch weiter erhöhen lässt. 