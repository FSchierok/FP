\section{Durchführung}
\label{sec:Durchführung}
\subsection{Aufbau}
Zuerst muss der Versuch aufgebaut werden. Dazu wird an den Tank ein Kreuzung K1 angebracht. An beide Abzweige dieser Kreuzung wird ein Sperrventil V1/V2
montiert und dann das Nadelventil bzw ein Schlauch S1. An den letzte Ausgang der Kreuzung K1 kommt ein T-Stück T1. An diesen Abzweig wird die Kaltkathoden
angebracht. Der letzte Ausgang von T1 geht in ein weiteres T-Stück T2, an dem die Glühkathode und hinter einem Sperrventil V3 die Turbopumpe angebracht sind.
An den Ausgang der Turbopumpe wird ein Sperrventil V4 und eine Kreuzung K2 montiert. An die Kreuzung K2 wird das andere Ende des Schlauches S1 angebracht. An K2
wird ebenfalls ein T-Stück T3 montiert, an dem das digitale und das analoge Piranimeter montiert sind. An den letzte Abzweig von K2 wird der Schlauch S2 und
damit die Drehschieberpumpe angebracht. Bei jedem Flansch ist darauf zu achten, dass eine passender Dichtungring eingesetzt wird. Dieser Aufbau ist auch
nochmal in den Abbildungen ?? dokumentiert.
